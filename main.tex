\documentclass[10pt, paper=a4,]{article}
\usepackage{graphicx}
%
%%%%%%%%%%%%%%%%%%%%%%%%%%%%%%%%%%%%%%%%%%%%%%%%%%%%%%%%%%%%%%%%%%%%%%%%%%%%
%   document style macros
%%%%%%%%%%%%%%%%%%%%%%%%%%%%%%%%%%%%%%%%%%%%%%%%%%%%%%%%%%%%%%%%%%%%%%%%%%%%
\def\Title#1{\begin{center} {\Large #1 } \end{center}}
\def\Author#1{\begin{center}{ \sc #1} \end{center}}
\def\Address#1{\begin{center}{ \it #1} \end{center}}
\def\andauth{\begin{center}{and} \end{center}}
\def\submit#1{\begin{center}Submitted to {\sl #1} \end{center}}
\newcommand\pubblock{\rightline{\begin{tabular}{l} Proceedings of DIS2022\\ 
         \pubdate  \end{tabular}}}

\newenvironment{Abstract}{\begin{quotation} \begin{center} 
             \large ABSTRACT \end{center}\bigskip 
      \begin{large}}{\end{large}\end{quotation}}

\newenvironment{Presented}{\begin{quotation} \begin{center} 
             PRESENTED AT\end{center}\bigskip 
      \begin{center}\begin{large}}{\end{large}\end{center} \end{quotation}}

\def\Acknowledgements{\bigskip  \bigskip \begin{center} \begin{large}
      \bf ACKNOWLEDGEMENTS \end{large}\end{center}}

%%%%%%%%%%%%%%%%%%%%%%%%%%%%%%%%%%%%%%%%%%%%%%%%%%%%%%%%%%%%%%%%%%%%%%%%%%%% 
%  personal abbreviations and macros
%    the following package contains macros used in this document:
%%%%%%%%%%%%%%%%%%%%%%%%%%%%%%%%%%%%%%%%%%%%%%%%%%%%%%%%%%%%%%%%%%%%%%%%%%%

\textwidth=6.5in
\textheight=8.75in
\hoffset=-0.85in
\voffset=-0.6in


% include packages you will need
\usepackage{color}
\usepackage{lineno}
\usepackage{hyperref}
\usepackage[style=phys,%
	biblabel=brackets,%
	chaptertitle=false,pageranges=false,%
	maxnames=4,sorting=none]{biblatex}
\usepackage[utf8]{inputenc}
\usepackage{amsmath,amsthm,amssymb}
\usepackage{graphicx}
\usepackage{physics}
\usepackage{subcaption}
\usepackage{tikz}
\usepackage{siunitx}
\usepackage{url}

\graphicspath{{images/}}
\addbibresource{references.bib}
%%%%%%%%%%%%%%%%%%%%%%%%%%%%%%%%%%%%%%%%%%%%%%%%%%%%%%%%%%%%%%%%%%%%
% basic data for the eprint:
%%%%%%%%%%%%%%%%%%%%%%%%%%%%%%%%%%%%%%%%%%%%%%%%%%%%%%%%%%%%%%%%%%%%


\newcommand\pubdate{\today}

%%  Affiliation
\def\affiliation{
On behalf of SeaQuest Collaboration, \\
Department of Physics, University of Illinois at Urbana-Champaign, \\Urbana, Illinois 61801, USA}

%% Acknowledge the support
\def\support{\footnote{Work supported by NSF}}


\newcommand{\conference}{DIS2022: XXIX International Workshop on Deep-Inelastic Scattering and Related Subjects\\
Santiago de Compostela, Spain\\
May 2-6 2022}

\begin{document}
\large
\begin{titlepage}
\pubblock


\vfill
\Title{Measurement of Charmonium Production in $p + p$ and $p + d$ Interaction in the Fermilab SeaQuest Experiment}
\vfill

\Author{Ching Him Leung \support}
\Address{\affiliation}
\vfill

\begin{Abstract}
SeaQuest has measured dimuon events from the interaction of \SI{120}{\GeV} proton beam on liquid hydrogen and deuterium targets with dimuon mass between \num{2} and \SI{8}{\GeV}. These dimuon events contain both the Drell-Yan process and the charmonium ($J/\psi$ and $\psi^\prime$) production. Unlike the Drell-Yan process which probes the antiquark distributions in the nucleons, the charmonium production is sensitive to both quark and gluon distributions. SeaQuest has extracted the $(p+d)/(p+p)$ cross section ratios as well as the differential cross sections for charmonium production in the kinematic region of $0.4 < x_F < 0.9$. The $(p+d)/(p+p)$ cross section ratios for charmonium production are found to be significantly different from that of the Drell-Yan process. The measured differential cross sections for charmonium production are compared with theoretical model calculations.
\end{Abstract}

\vfill

\begin{Presented}
\conference
\end{Presented}
\vfill
\end{titlepage}

\setcounter{footnote}{0}
\normalsize 


\section{Introduction}
\label{intro}

\section{E906/SeaQuest Experiment}

\section{Extraction of $J/\psi$ cross section}


\section{Conclusions}


\printbibliography[heading=bibintoc,title={References}]
\end{document}

